%%%%%%%%%%%%%%%%%
% This is an sample CV template created using altacv.cls
% (v1.1.2, 1 February 2017) written by LianTze Lim (liantze@gmail.com). Now compiles with pdfLaTeX, XeLaTeX and LuaLaTeX.
%
%% It may be distributed and/or modified under the
%% conditions of the LaTeX Project Public License, either version 1.3
%% of this license or (at your option) any later version.
%% The latest version of this license is in
%%    http://www.latex-project.org/lppl.txt
%% and version 1.3 or later is part of all distributions of LaTeX
%% version 2003/12/01 or later.
%%%%%%%%%%%%%%%%

%% If you need to pass whatever options to xcolor
\PassOptionsToPackage{dvipsnames}{xcolor}

%% If you are using \orcid or academicons
%% icons, make sure you have the academicons
%% option here, and compile with XeLaTeX
%% or LuaLaTeX.
% \documentclass[10pt,a4paper,academicons]{altacv}

%% Use the "normalphoto" option if you want a normal photo instead of cropped to a circle
% \documentclass[10pt,a4paper,normalphoto]{altacv}

\documentclass[10pt,a4paper]{altacv}

%% AltaCV uses the fontawesome and academicon fonts
%% and packages.
%% See texdoc.net/pkg/fontawecome and http://texdoc.net/pkg/academicons for full list of symbols.
%%
%% Compile with LuaLaTeX for best results. If you
%% want to use XeLaTeX, you may need to install
%% Academicons.ttf in your operating system's font
%% folder.


% Change the page layout if you need to
\geometry{left=1cm,right=9cm,marginparwidth=6.8cm,marginparsep=1.2cm,top=1.25cm,bottom=1.25cm}

% Change the font if you want to.

% If using pdflatex:
\usepackage[utf8]{inputenc}
\usepackage[T1]{fontenc}
\usepackage[default]{lato}

% If using xelatex or lualatex:
% \setmainfont{Lato}

% Change the colours if you want to
\definecolor{Blue}{HTML}{0a4399}
\definecolor{DarkBlue}{HTML}{08316e}
\definecolor{SlateGrey}{HTML}{2E2E2E}
\definecolor{LightGrey}{HTML}{666666}
\colorlet{heading}{DarkBlue}
\colorlet{accent}{Blue}
\colorlet{emphasis}{SlateGrey}
\colorlet{body}{LightGrey}

% Change the bullets for itemize and rating marker
% for \cvskill if you want to
\renewcommand{\itemmarker}{{\small\textbullet}}
\renewcommand{\ratingmarker}{\faCircle}

%% sample.bib contains your publications
\addbibresource{sample.bib}

\begin{document}
\name{Joseph Thomas C. Miclat}
\tagline{Data Scientist and Software Engineer}
\photo{3cm}{myfoto}
\personalinfo{%
  % Not all of these are required!
  % You can add your own with \printinfo{symbol}{detail}
  \email{jtmiclat@gmail.com}
%  \phone{000-00-0000}
%  \mailaddress{Quezon City, Metro Manila, Phillip}
%  \location{Quezon City, Philippines}
%  \homepage{jtmiclat.com}
%  \twitter{@Jtmiclat}
  \linkedin{linkedin.com/in/jt-miclat-410444143}
  \github{github.com/jtmiclat}
  %% You MUST add the academicons option to \documentclass, then compile with LuaLaTeX or XeLaTeX, if you want to use \orcid or other academicons commands.
%   \orcid{orcid.org/0000-0000-0000-0000}
}

%% Make the header extend all the way to the right, if you want.
\begin{fullwidth}
\makecvheader
%Life is like a box of chocalate
\end{fullwidth}

%% Provide the file name containing the sidebar contents as an optional parameter to \cvsection.
%% You can always just use \marginpar{...} if you do
%% not need to align the top of the contents to any
%% \cvsection title in the "main" bar.

\cvsection[Resume-p1sidebar]{About Me}
Aspiring Data Scientist and Software Engineer. I am passionate about Data Science and Software Development.

\cvsection{Work Experience}

\cvevent{Data Scientist and Software Engineer}{Pez.AI}{August 2017 -- Present}{Mandaluyong City, Philippines}
\begin{itemize}
\item Implementing Sentiment, Named Entity, and Document Similarity models
\item Implementing Client facing Analytic dashboards
\item Creating ETL pipelines for bot/chat data
\item Developing and deploying messaging services
\item Managing Kubernetes deployments under Google Cloud Platform
\end{itemize}

\divider

\cvevent{Student Research Assistant}{Manila Observatory}{March 2015 -- May 2017}{Quezon City, Philippines}
\begin{itemize}
\item Researched on Regional Climate Models in the context of the Philippines
\item Determined the difference between Hydrostatic and Nonhydrostatic core parametization on the Philippine Monsoon Season
\end{itemize}

\divider

\cvevent{On the Job Trainee}{Philippine Nuclear Research Institute}{May 2016 -- July 2016}{Quezon City, Philippines}
\begin{itemize}
\item Collecting and Analyzing Black Carbon Samples
\item Determining if local Black Carbon spikes are correlated to trans-boundary pollution
\item Measuring Gamma Radioactivity of different consumables
\end{itemize}

\cvsection{Workshops and Experiences}

\cveventshort{Google DevFest Philippines}{October 14, 2017}{Philippines}
\smallskip

\cveventshort{TensorFlow Dev Summit Extended Manila}{April 14, 2017}{Philippines}
\smallskip

\cveventshort{Regional Workshop on identifying Trans-Boundary Air Pollution Events Across Asia-Pacific}{June 27 -- July 1 , 2016}{Philippines}
\smallskip

\cveventshort{Regional Climate Models (RegCM) Training Workshop for Southeast Asia}{May 25 -- 29, 2015}{Philippines}
\smallskip

\cveventshort{First Cern School Philippines 2014}{May 31 -- April 8, 2014}{Philippines}
\smallskip


\end{document}
